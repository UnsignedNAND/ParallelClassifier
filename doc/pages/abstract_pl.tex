\chapter*{Streszczenie}
\addcontentsline{toc}{chapter}{Streszczenie}

Mimo wielu prowadzonych badań klasyfikacja dokumentów tekstowych pozostaje nadal skomplikowanym problemem. Wraz ze wzrastającą potrzeba archiwizacji danych wzrosła również konieczność umożliwiania szybkiego i automatycznego dostępu do skategoryzowanych informacji. Obecnie klasyfikacja dokumentów tekstowych jest używana w wielu profesjonalnych dziedzinach. Algorytmy eksploracji danych odgrywają kluczową rolę nie tylko w zastosowaniach biznesowych czy też aplikacjach naukowych, ale również jako narzędzia diagnostyczne wspierające decyzje podczas leczenia.

Motywacją dla tej pracy było stworzenie skalowalnego systemu udostępniającego algorytmy dla wybierania cech z dokumentów tekstowych, ich grupowania przy użyciu algorytmu klasteryzacji, oraz klasyfikowania nowych dokumentów. Implementacja algorytmów była kierowana przystosowaniem ich do działania w rozproszeniu na wielu procesach. Wymienione cele zostały opracowane, zaimplementowane oraz przetestowane pod kątem czasu przetwarzania i specyficzności zdefiniowanej jako procent poprawnie sklasyfikowanych dokumentów spośród wszystkich klasyfikowanych dokumentów.

Wybór cech reprezentujących dokument jest kluczowym czynnikiem wpływającym na jakość dalszego przetwarzania dokumentu. Dokumenty składają się ze zdań, słów, notek łączących artykuł jeden z drugim tworząc relacje między nimi. Ważnym zagadnieniem towarzyszącym temu etapowi jest redukcja wymiarowości danych, czyli określenie które cechy są istotne, a które wprowadzają niepotrzebny szum. Bazując na tej wiedzy można określić stopień podobieństwa między dokumentami, tworząc ich grupy.

W celu klasteryzacji zastosowano wielowątkową implementację algorytmu k-means. Algorytm ten został wybrany ze względu na liniową złożoność czasu względem ilości dokumentów oraz możliwość rozproszenia pracy pomiędzy procesami roboczymi. Pierwszym krokiem w algorytmie k-means jest wybranie \textit{k} centroidów, które posłużą jako punkty odniesień podczas obliczeń przynależności. Następnie obliczana jest odległość będąca miarą podobieństwa między dokumentami, co pozwala na przypisanie podobnych dokumentów do tej samej grupy. Na bazie nowo stworzonych grup można wyznaczyć nowe środki grup. Zwykle centroidy wybierane są w sposób losowy, co może przekładać się na inny wynik końcowy.

Następnym etapem jest klasyfikacja, dla której zaproponowany został dwuetapowy klasyfikator. Klasyfikacja to proces przypisywania etykiety dla nowego dokumentu w oparciu o wiedzę zgromadzona podczas wcześniejszego etapu klasteryzacji. Klasyfikacja zakłada, że dokumenty są pogrupowane oraz oznaczone etykietą. Pierwsza warstwa klasyfikacji składa się z klasyfikatora odległościowego kNN, którego wyniki przekazywane są do dalszej analizy przez drugą warstwę z klasyfikatorem SVM. Obie metody zostały zaprojektowane i zoptymalizowane do wykorzystywania wieloprocesorowych systemów. Klasyfikator kNN odpowiedzialny za zawężenie zbioru klas do których dokument może zostać przypisany. Algorytm SVM, bazując na zawężonym zbiorze, podejmuje precyzyjną decyzję do jakiej klasy dany dokument powinien należeć.

Przeprowadzone testy pozwoliły określić wpływ zmiany parametrów na ostateczny wynik działania systemu. Zmiana parametru \textit{k} dla klasyfikatora kNN z wartości 10 do 40 pozwoliła uzyskać lepszą specyficzność o 8% dla kNN oraz 12% dla SVM. Zmierzony również został czas działania powyższych metod przy zmianie parametru – przy metodzie kNN nie zauważono wpływu na czas działania przy zmianie parametru, lecz dla metody SVM zmiana wpływała niekorzystnie na czas działania wydłużając go nawet 21 razy w zaprezentowanym przykładzie. Przeprowadzone badania parametru \textit{k} dla algorytmu k-means wykazały, iż jego zwiększenie nieznacznie wpływa na poprawę specyficzności zarówno dla kNN jak i SVM. 

W pracy zaprezentowano także główne ograniczenia omówionych metod. Obejmują one czas potrzebny do komunikacji i przesyłania danych między procesami. Można zauważyć, iż aplikacja osiąga największą wydajność dla 6 procesów wykonawczych. Używanie większej ilości procesów wykonawczych skutkuje zwiększeniem czasu koniecznego dla komunikacji oraz wymiany danych, co prowadzi do stałego czasu działania aplikacji. Sposoby eliminacji opisanych problemów zostały również przedyskutowane.

Wstępne wyniki wskazują, iż klasyfikacja nowego dokumentu przy zbiorze 100 jest możliwa w satysfakcjonującym czasie (6 sekund). Dodatkowo, uzyskane rezultaty skalują się niemal liniowo dla wzrastającej liczby dokumentów przy systemach 10 korowych. Uzyskane wyniki pozwalają przypuszczą, iż zaprezentowane rozwiązanie może byś częścią większej platformy, która klasyfikuje dokumentu w istniejących systemach. Jednakże uzyskane rezultaty są jedynie wstępne i powinny być dalej przebadane i przetestowane.


\vspace{12pt}
\noindent\textbf{Keywords:} klasyfikacja dokumentów, klasteryzacja dokumentów, kNN, SVM, k-means

\vspace{12pt} \noindent\textbf{Dziedzina nauki i techniki, zgodnie z wymogami
	OECD}: \textless{}1.2 Nauki o komputerach i informatyka\textgreater{}