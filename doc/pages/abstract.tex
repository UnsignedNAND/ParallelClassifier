\chapter*{Abstract}
\addcontentsline{toc}{chapter}{Abstract}

The problem of text document categorization has became more complex than ever. With continuously increasing needs for storing huge amounts of data, need for quick access to data and automatic association also rises. Nowadays, document classification is used in wide range of application, varrying from knowledge databases, medical applications and spam filtering.

In this work algorithms of classification in parallel environment were designed, implemented and tested for processing time and recall defined as percentage of properly classified documents to all documents. 

For clusterization a parallel implementation of k-means algorithm was introduced. The choice of \textit{k} parameter and its influence on correctness of final results was also tested. Next, a multi-tier classification method was designed. First layer of classification consisted of kNN classifier, while second was based on SVM. Both methods were adapted to take advantage of available multi-core systems. kNN classifier was responsible for initial narrowing of possible labels for new document. Next, SVM algorithm was utilized to define final document class.

The main limitations of all methods were also highlighted in this work. They include time needed to communicate and send data between processes. The ways to overcome them are also discussed.

Initial results show that it is possible to classify a new document in satisfactory time interval of 6 seconds for a hundred of documents and scales nearly linearly on 10 core system. These values are indicators for the assumptions that this system can be a part of bigger platforms that cope with document classifications in real use cases. However, the results are only preliminary and should further tested in the future researches.